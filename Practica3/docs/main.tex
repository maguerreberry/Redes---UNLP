%++++++++++++++++++++++++++++++++++++++++
% Don't modify this section unless you know what you're doing!
\documentclass[letterpaper,12pt]{article}
\usepackage[utf8]{inputenc}
\usepackage{float}
\usepackage{tabularx} % extra features for tabular environment
\usepackage{amsmath}  % improve math presentation
\usepackage{graphicx} % takes care of graphic including machinery
\usepackage[margin=1in,letterpaper]{geometry} % decreases margins
\usepackage{cite} % takes care of citations
\usepackage[final]{hyperref} % adds hyper links inside the generated pdf file
\usepackage[table,xcdraw]{xcolor}
\hypersetup{
	colorlinks=true,       % false: boxed links; true: colored links
	linkcolor=blue,        % color of internal links
	citecolor=blue,        % color of links to bibliography
	filecolor=magenta,     % color of file links
	urlcolor=blue         
}
%++++++++++++++++++++++++++++++++++++++++


\begin{document}

\title{Práctica 3 - Ruteo}
\author{Matthew Aguerreberry, Natasha Tomattis}
\date{\today}
\maketitle

% \begin{abstract} 
% \end{abstract}


\section{Practica de Ruteo - OSPF}
	\begin{figure}[ht] 
			
		\centering \includegraphics[width=0.8\columnwidth]{figure/topo_figura1.png}
		%\includegraphics[width=1.0\columnwidth]{sr_setup.pdf}
		\caption{
				\label{fig:samplesetup} % spaces are big no-no 
				Implementación Ejercicio 1.
		}
	\end{figure}
	\begin{enumerate}
		\item \textbf{Conectar los routers, switchs y nodos segun el diagrama de la figura 1.}
		\item \textbf{Configurar interfaces de acuerdo al diagrama.}
		\item \textbf{¿Configurar OSPF en todas los routers respetando las areas segun el diagrama. Es posible alcanzar todas las redes?} \\
		Si, todas las redes son compartidas a traves de OSPF.
		\item \textbf{Los routers de las areas no backbone, conocen todas las rutas a las demas redes?}\\
		Si, la division de areas no limita la propagacion de rutas OSPF.
		\item \textbf{Observando el área backbone, que router fue elegido como DR y cual como BDR? ¿Por qué fueron elegidos esos routers?}\\
		Se eligio el router con la router id más alto de la red 10.0.10.0/24 como DR y el subsiguiente como BDR. El router ID, por defecto, toma el valor de IP más bajo de las interfaces activas del dispositivo.
		\item \textbf{Elegir uno de los routers que no cumple una de esas funciones y configurarlo para que sea DR (no debe modificar el direccionamiento) ¿De qué manera/s puede hacer esto? Seleccione una para lograr lo solicitado.}\\
		Esto se puede lograr de varias maneras, por un lado se puede cambiar la prioridad del router la cual sera utilizada a la hora de la eleccion del DR; por otro lado se puede modificar el router-id ya sea a traves de la configuracion explicita de este parametro o la configuracion de una interfaz loopback con una direccion de mayor valar que los otros router-id. En nuestro caso configuramos una interfaz loopback1 en el dispositivo 2003 para que esta sea utilizada como router-id, como se puede ver en la imagen a continuacion.
		\begin{figure}[H] 
			
			\centering \includegraphics[width=0.5\columnwidth]{figure/int_loopback.png}
			%\includegraphics[width=1.0\columnwidth]{sr_setup.pdf}
			\caption{
					\label{fig:samplesetup} % spaces are big no-no 
					Configuracion de la interfaz loopback en el router 2003.
			}
		\end{figure} 
		\begin{figure}[H] 
			
			\centering \includegraphics[width=0.8\columnwidth]{figure/new_dr.png}
			%\includegraphics[width=1.0\columnwidth]{sr_setup.pdf}
			\caption{
					\label{fig:samplesetup} % spaces are big no-no 
					Seleccion de la IP configurada en loopback como roter-id. Eleccion de nuevo DR.
			}
		\end{figure}
		\item \textbf{¿Por qué se recomienda definir una dirección de loopback cuando se
		utiliza OSPF?}\\
		Para poder controlar la selección del router-id, y por consiguiente la selección del router designado. Ya que si no se configura una interfaz loopback, esta desición depende de las interfaces configuradas, las cuales pueden cambiar dinamicamente.
		\item \textbf{Explique los distintos tipos de áreas en OSPF}\\
		\item \textbf{En las areas no backbone, ¿también se elige un DR y un BDR?}\\
		Sí.
		\item \textbf{Definir el área 10 como area stub. ¿Qué sucede con la tabla de ruteo
		en el router 2001?}\\
		Se le agraga un default gateway, ya que establecer un área stub impide que el área conozca de rutas externas a OSPF, ete permite el acceso a rutas externas si las hubiera a traves de 2003(ABR). En este caso como todas las rutas aprendidas son por OSPF, no se eliminan entradas en la tabla.
		\begin{figure}[H] 
			
			\centering \includegraphics[width=0.7\columnwidth]{figure/routes_default_area.png}
			%\includegraphics[width=1.0\columnwidth]{sr_setup.pdf}
			\caption{
					\label{fig:samplesetup} % spaces are big no-no 
					Rutas en 2001 antes de la configuración como stub area.
			}
		\end{figure} 
		\begin{figure}[H] 
			
			\centering \includegraphics[width=0.7\columnwidth]{figure/routes_stub_area.png}
			%\includegraphics[width=1.0\columnwidth]{sr_setup.pdf}
			\caption{
					\label{fig:samplesetup} % spaces are big no-no 
					Rutas en 2001 después de la configuración como stub area.
			}
		\end{figure}
		\item \textbf{¿Cómo definiría el área 10 para que 2001 solo aprenda de 2003 una
		ruta default?}\\
		Esto se puede lograr configurando el área 10 como \textit{totally stubby area}, este tipo de configuración impide que se intercambien LSAs de tipo 3 que contienen información sobre las rutas de las otras áreas. El comando utilizado en este caso fue \textit{"/routing ospf area set numbers=1 type=stub inject-summary-lsa=no"}
		\begin{figure}[H] 
			
			\centering \includegraphics[width=0.7\columnwidth]{figure/routes_totally_stub.png}
			%\includegraphics[width=1.0\columnwidth]{sr_setup.pdf}
			\caption{
					\label{fig:samplesetup} % spaces are big no-no 
					Rutas en 2001 después de la configuración como totally stubby area.
			}
		\end{figure}
		\item \textbf{¿Qué mensaje envía el router seleccionado en el área backbone? ¿A
		quién se lo envía? ¿Qué dirección utiliza? ¿Qué hace el receptor de
		este mensaje?}\\
		El router envia un mensaje LSA Update, con un LSA de tipo 1 (Router Update) a la dirección multicast 224.0.0.6 (Grupo conformado por DR y BDR). El DR envia el mismo mensaje a la dirccion multicast 224.0.0.5 (Grupo conformado por todos los routers), para que actualicen sus bases de datos.
		\begin{figure}[H] 
			
			\centering \includegraphics[width=0.7\columnwidth]{figure/2004_ether2_down.png}
			%\includegraphics[width=1.0\columnwidth]{sr_setup.pdf}
			\caption{
					\label{fig:samplesetup} % spaces are big no-no 
					Mensajes en el area backbone luego de deshabilitar una interfaz.
			}
		\end{figure}
		\item \textbf{¿Cómo calcula OSPF el mejor camino a una red? ¿Qué algoritmo
		utiliza?}\\
		El cálculo del mejor camino es en base al ancho de banda de los enlaces, utilizando el alrgoritmo de Dijkstra.
	\end{enumerate}

	\subsection{Enlaces consultados}
		\begin{itemize}
			\item{HCNA Networking Study Guide}  \\
			\textit{Springer. Huawei Technologies Co., Ltd.},Ch 8.2 RIP.
			\item{Understanding OSPF Stub Areas, Totally Stubby Areas, and Not-So-Stubby Areas}  \\
			\url{https://www.juniper.net/documentation/en_US/junos/topics/concept/ospf-stub-areas-overview.html}
			\item{Manual:Routing/OSPF}  \\
			\url{https://wiki.mikrotik.com/wiki/Manual:Routing/OSPF}
		\end{itemize}
\end{document}